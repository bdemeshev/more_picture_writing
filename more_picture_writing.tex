\documentclass{article}



\usepackage{arxiv}

\usepackage[utf8]{inputenc} % allow utf-8 input
\usepackage[T1]{fontenc}    % use 8-bit T1 fonts
\usepackage{hyperref}       % hyperlinks
\usepackage{url}            % simple URL typesetting
\usepackage{booktabs}       % professional-quality tables
\usepackage{amsfonts}       % blackboard math symbols
\usepackage{nicefrac}       % compact symbols for 1/2, etc.
\usepackage{microtype}      % microtypography
\usepackage{lipsum}		% Can be removed after putting your text content
\usepackage{graphicx}
\usepackage{natbib}
\usepackage{doi}

\usepackage{tikz}
\usepackage{amsmath}

\DeclareMathOperator{\E}{\mathbb{E}}
\newcommand{\up}{\uparrow}
\newcommand{\down}{\downarrow}
\newcommand{\bigb}{B}
\newcommand{\babyb}{b}


\title{More picture writing}

%\date{September 9, 1985}	% Here you can change the date presented in the paper title
%\date{} 					% Or removing it

\author{ \href{https://orcid.org/0000-0003-2284-3204}{\includegraphics[scale=0.06]{orcid.pdf}\hspace{1mm}Boris~Demeshev}\\ %\thanks{github.com/bdemeshev/me} \\
	Department of Computer Science, Department of Applied Economics\\
	State University Higher School of Economics\\
	Moscow \\
	\texttt{bdemeshev@hse.ru} \\
	%% examples of more authors
	%% \AND
	%% Coauthor \\
	%% Affiliation \\
	%% Address \\
	%% \texttt{email} \\
	%% \And
	%% Coauthor \\
	%% Affiliation \\
	%% Address \\
	%% \texttt{email} \\
	%% \And
	%% Coauthor \\
	%% Affiliation \\
	%% Address \\
	%% \texttt{email} \\
}

% Uncomment to remove the date
%\date{}

% Uncomment to override  the `A preprint' in the header
%\renewcommand{\headeright}{Technical Report}
%\renewcommand{\undertitle}{Technical Report}
\renewcommand{\shorttitle}{More picture writing}

%%% Add PDF metadata to help others organize their library
%%% Once the PDF is generated, you can check the metadata with
%%% $ pdfinfo template.pdf
\hypersetup{
pdftitle={More picture writing},
pdfsubject={q-bio.NC, q-bio.QM},
pdfauthor={Boris~Demeshev},
pdfkeywords={First keyword, Second keyword, More},
}

\begin{document}
\maketitle

\begin{abstract}
	In this article we give more picture writing examples that motivate and explain generating functions. 
\end{abstract}


\section{Picture writing approach}

The picture writing approach to motivate generating functions was introduced in an old article \cite{polya1956picture}.
The approach is visually appealing and really explains \textit{why} generating functions just count the objects of interest.

Despite its beauty the picture writing approach is almost unknown. So we'll try to provide more examples!


\section{Fibonacci sequence}


The rules to produce the Fibonacci sequence are simple. 
Start with one young pair of rabbits, $F_1 = 1$. 
Each young pair will be adult in one month, each adult pair will give birth to a new young pair every month. 
The number of pairs alive at time $n$ is $F_n = F_{n-1} + F_{n-2}$, 
where $F_{n-2}$ pairs are alive at $(n-2)$, so they are 
adult at $(n-1)$, hence they give birth to new pairs at $n$. 
Let's draw a picture a little bit like a pedigree tree:


!!!! Here I need a picture

Every living pair of rabbits at every moment of time may be identified with a pedigree string.
For example there are three strings of length four: $\babyb\bigb\bigb\bigb$, 
$\babyb\bigb\bigb\babyb$, $\babyb\bigb\babyb\bigb$.

Let $S$ be the set of all valid pedigrees. 
A pedigree is valid if it starts with baby bunny $\babyb$ and every baby bunny $\babyb$ is followed by big bunny $\bigb$. 

\[
S = \{ \babyb, \babyb\bigb, \babyb\bigb\bigb, \babyb\bigb\babyb, \ldots \} 	
\]

Instead of writing a set with curly brackets and comma we will use the plus symbol $+$.


\[
S =  \babyb + \babyb\bigb + \babyb\bigb\bigb +  \babyb\bigb\babyb + \ldots
\]

We introduce multiplication of pedigrees. 
To multiply two pedigrees we just write them one next to another. 
For example

\[
\babyb\bigb\bigb \cdot 	\babyb\bigb = \babyb\bigb\bigb\babyb\bigb
\]

The multiplication of two valid pedigrees may give rise to an unvalid pedigree,
as here

\[
   \babyb\bigb\bigb \babyb \cdot \babyb \bigb = \babyb\bigb\bigb \babyb\babyb \bigb.
\]


Let's move on to the list of valid pedigrees. 
Each valid pedigree with two or more symbols ends with $\bigb$ or with $\bigb\babyb$. 
We take out the corresponding terms:


\[
S =  \babyb + (\babyb + \babyb\bigb + \babyb\bigb\bigb +  \babyb\bigb\babyb + \ldots) \cdot \bigb + (\babyb + \babyb\bigb + \babyb\bigb\bigb +  \babyb\bigb\babyb + \ldots) \cdot \bigb\babyb
\]

Or simply 

\[
S = \babyb + S\bigb + S\bigb\babyb	
\]

Now we just simplify our notation. 
Imagine we don't care whether the bunny is young or adult and we just write $t$ for both types. 

By definition $g(\bigb) = t$, $g(\babyb\bigb\babyb) = t^3$ and $g(\babyb\bigb + \bigb\babyb +\bigb) = 2t^2 + t$.
The polynomial $g(A) = t^3 + 2t^{15}$ means that 
the set $A$ contains one pedigree with three bunnies and two pedigrees with fifteen bunnies.

In formal world the function $g$ is a homomorphism between sets of bunny pedigrees and polynomials. 
The function $g$ preserves addition $+$ and multiplication $\cdot$.

Hence we obtain the recurrence relation for the generating function
\[
g(S) = t + t g(S) + t^2 g(S)	
\]

Hence
\[
g(S) = \frac{t}{1 - t - t^2}	
\]








It's funny to draw bunnies on blackboard, they certainly catch student's attention. 
To simplify drawing one may just write B for Big bunny and b for baby bunny. 


\section{Paths to win one dollar}

Imagine that I bet one dollar on a fair coin toss. My fortune may go up $\up$ or down $\down$ in every bet.
I bet until I will increase my wellfare by one dollar. Let's count all the possible paths!

For example the set of possible possible paths $S$ contains the path $\up$ and the path $\down\up\up$ but not the path $\up\down\down\up\up$. 
The path $\up\down\down\up\up$ is not included in $S$ because I will stop after the first $\up$. 

Hence
\[
S = \{ \up, \down\up\up, \down\up\down\up\up, \down\down\up\up\up, \ldots \}	
\]

Instead of writing curly brackets and comma to denote a set we will just use the plus sign $+$.

\[
S =  \up + \down\up\up + \down\up\down\up\up + \down\down\up\up\up + \ldots 	
\]

We don't make the difference between one sequence, say $\down\up\up$, and 
a set containing one sequence, $\{\down\up\up \}$. 
When writing one the wall you may omit curly brackets! It's ok!


We need to invent multiplication of paths. That's easy! Just write two sequences next one to another.
By this definition the multiplication is non-commutative:
\[
\up\down \cdot \down \down \up = \up\down \down \down \up  \neq \down \down \up \up\down =  \down \down \up \cdot \up\down 
\]

The multiplication of two sets or more sets is defined elementwise:
For example
\[
	\up\down \cdot (\down \down \up + \down) = \up\down\down \down \up + \up\down \down
\]

Now we are ready to state the main recurrent equation for our set $S$. 


If I loose the first dollar then I need to win one dollars twice!
\[
S = \up + \down \cdot S \cdot S	
\]

This is exactly the equation for generating functions! 
We just simplify our notation. 
Now we don't care whether the move is up or down and insted of both $\up$ and $\down$ we just write $t$. 

By definition $g(\up) = t$, $g(\up\up\down) = t^3$ and $g(\up\up + \down\down +\up) = 2t^2 + t$.
The polynomial $g(A) = t^3 + 2t^{15}$ means that 
the set $A$ contains one sequence with three arrows and two sequences with fifteen arrows.

In formal world the function $g$ is a homomorphism between sets of up or down sequencies and polynomials. 
The function $g$ preserves addition $+$ and multiplication $\cdot$.

Hence we obtain the recurrence relation for the generating function
\[
g(S) = t + t g(S) g(S)	
\]

Now the standard machine of generating function may be used. 
I reproduce it here for completeness of exposition. 

There are two solutions of the quadratic equation
\[
g(S) = \frac{1\pm \sqrt{1 - 4t^2}}{2t}	
\]

Instead of boring differentiation in the Taylor expansion formula we use Newton's approach
\[
(1 + x)^a = \binom{a}{0} + \binom{a}{1} x + \binom{a}{1} x^2 + \ldots 
\]

Here $\binom{a}{0} = 1$ and for $k>0$:
\[
\binom{a}{k} = \frac{a(a-1)(a-2)\cdot \ldots \cdot (a-k + 1)}{k!} 	
\]

Hence 
\[
g(S)  = \frac{1\pm \left(1 + 0.5 (-4t^2) + \frac{0.5(0.5- 1)}{2!}(-4t^2)^2 + \frac{0.5(0.5 - 1)(0.5 - 2)}{3!}(-4t^2)^3 +\ldots    \right)}{2t}	
\]

All coefficients in $g(S)$ should be non-negative as they just counts the number of corresponding sequences.
Ruling out this case we should leave the $-$ in nominator. 

Finally we get
\[
g(S) = \sum_{k=0}^{\infty} \binom{0.5}{k} (-1)^k 2^{2k - 1} t^{2k - 1} = t + t^3 + 2t^5 + \ldots 	
\]


So the number of paths to win one dollar exactly in $2k - 1$ tosses is 
\[
	c_{2k - 1} = \binom{0.5}{k} (-1)^k 2^{2k - 1}.
\]
It can be simplified to
\[
  c_{2k - 1} = \frac{2^{k-1}\cdot 1\cdot 3 \cdot 5 \cdot \ldots (2k - 3)}{k!}	
\]




\section{Catalan numbers}

The representation of Catalan numbers as triangulations of convex $n$-gon 
was considered in \cite{polya1956picture}. 
So I will consider the representation of Catalan numbers with bracket sequences.

The set of all valid brackets sequences including empty sequence $\square$ is 
\[
S = \square + () + (()) + ()() + ((())) + (())() + ()(()) + \ldots
\]

Multiplication is intuitive and non-commutative 
\[
 () \cdot (()) = ()(())
\]

Let's focus on the term that is inside the leftmost opening bracket ( and its closing sister ).

This term can be any valid bracket sequence $S$. 
To the right of them we can also have any valid bracket sequence $S$. 

Hence we obtain the decomposition 
\[
S = \square + (\cdot S \cdot )\cdot S
\]

Each bracket pair is replaced by $t$. 
The multiplication of bracket sequences is non-commutative but the multiplication of $t$-polynomials is commutative. 
Empty bracket sequence with no brackets is replaced by $t^0 = 1$.

And we get the equation 
\[
g(S) = 1 + 	t g(S) g(S)
\]

Then the standard generating function technique may be used. 


\section{Discussion}


This picture writing approach may be used to introduce abstract algebra to college students. 
We multiply sequences of pictures or just words instead of numbers. 
We write equations where the solution is a set of picture writings. 
That's more abstract and more fun!

Formally speaking all sets of picture writings for the given problem is not a ring. 
There are no inverse element for addition as addition is the union of sets. 
Even more, there is an ambiguity in the definition of with a sum of a picture sequence with itself. 
We avoid this problem by never adding the same picture sequence to itself. 

To resolve the problem one may keep record of the number of each picture sequence.
Like $S = 2 ab + abb + 3 abcd$. In this case $g(S) = 2t^2 + t^3 + 3t^4$.

To introduce equations on words sets to college students one may start with a tribe 
that has only one letter $a$ in the alphabet. 
All the available words including silence $\square$ are

\[
S = \square + a + aa + aaa + aaaa + \ldots	
\]

The dictionary of this tribe satisfies the equation
\[
S = \square + a \cdot S	
\]

If there are two letters in the alphabet the equation is 

\[
S = \square + a \cdot S + b \cdot S	
\]

And let the fun begin! 


% keywords can be removed
% \keywords{First keyword \and Second keyword \and More}



\bibliographystyle{unsrtnat}
\bibliography{references}  %%% Uncomment this line and comment out the ``thebibliography'' section below to use the external .bib file (using bibtex) .


%%% Uncomment this section and comment out the \bibliography{references} line above to use inline references.
% \begin{thebibliography}{1}

% 	\bibitem{kour2014real}
% 	George Kour and Raid Saabne.
% 	\newblock Real-time segmentation of on-line handwritten arabic script.
% 	\newblock In {\em Frontiers in Handwriting Recognition (ICFHR), 2014 14th
% 			International Conference on}, pages 417--422. IEEE, 2014.

% 	\bibitem{kour2014fast}
% 	George Kour and Raid Saabne.
% 	\newblock Fast classification of handwritten on-line arabic characters.
% 	\newblock In {\em Soft Computing and Pattern Recognition (SoCPaR), 2014 6th
% 			International Conference of}, pages 312--318. IEEE, 2014.

% 	\bibitem{hadash2018estimate}
% 	Guy Hadash, Einat Kermany, Boaz Carmeli, Ofer Lavi, George Kour, and Alon
% 	Jacovi.
% 	\newblock Estimate and replace: A novel approach to integrating deep neural
% 	networks with existing applications.
% 	\newblock {\em arXiv preprint arXiv:1804.09028}, 2018.

% \end{thebibliography}


\end{document}
